%%
%% This is file `sample-sigconf.tex',
%% generated with the docstrip utility.
%%
%% The original source files were:
%%
%% samples.dtx  (with options: `sigconf')
%% 
%% IMPORTANT NOTICE:
%% 
%% For the copyright see the source file.
%% 
%% Any modified versions of this file must be renamed
%% with new filenames distinct from sample-sigconf.tex.
%% 
%% For distribution of the original source see the terms
%% for copying and modification in the file samples.dtx.
%% 
%% This generated file may be distributed as long as the
%% original source files, as listed above, are part of the
%% same distribution. (The sources need not necessarily be
%% in the same archive or directory.)
%%
%% The first command in your LaTeX source must be the \documentclass command.
\documentclass[sigconf]{acmart}

%%
%% \BibTeX command to typeset BibTeX logo in the docs
\AtBeginDocument{%
  \providecommand\BibTeX{{%
    \normalfont B\kern-0.5em{\scshape i\kern-0.25em b}\kern-0.8em\TeX}}}

%% Rights management information.  This information is sent to you
%% when you complete the rights form.  These commands have SAMPLE
%% values in them; it is your responsibility as an author to replace
%% the commands and values with those provided to you when you
%% complete the rights form.
\setcopyright{acmcopyright}
\copyrightyear{2019}
\acmYear{2019}
\acmDOI{}

%% These commands are for a PROCEEDINGS abstract or paper.
\acmConference[CASCON '19]{CASCON '19: 29th Annual International Conference on Computer Science and Software Engineering}{November 04--06, 2019}{Markham, Canada}
\acmBooktitle{CASCON '19: CASCON, November 04--06, 2019, Markham, Ontario, Canada}
\acmPrice{0.00}
\acmISBN{}


%%
%% Submission ID.
%% Use this when submitting an article to a sponsored event. You'll
%% receive a unique submission ID from the organizers
%% of the event, and this ID should be used as the parameter to this command.
%%\acmSubmissionID{123-A56-BU3}

%%
%% The majority of ACM publications use numbered citations and
%% references.  The command \citestyle{authoryear} switches to the
%% "author year" style.
%%
%% If you are preparing content for an event
%% sponsored by ACM SIGGRAPH, you must use the "author year" style of
%% citations and references.
%% Uncommenting
%% the next command will enable that style.
%%\citestyle{acmauthoryear}

%%
%% end of the preamble, start of the body of the document source.
\begin{document}

%%
%% The "title" command has an optional parameter,
%% allowing the author to define a "short title" to be used in page headers.
\title{Compiler-Driven Performance Workshop}

%%
%% The "author" command and its associated commands are used to define
%% the authors and their affiliations.
%% Of note is the shared affiliation of the first two authors, and the
%% "authornote" and "authornotemark" commands
%% used to denote shared contribution to the research.
\author{Clark Verbrugge}
%\authornote{Both authors contributed equally to this research.}
%\orcid{1234-5678-9012}
%\author{}
%\authornotemark[1]
%\email{}
\affiliation{%
  \institution{McGill University}
  \streetaddress{}
  \city{Montr\'{e}al}
  \state{Quebec}
  \country{Canada}
}
\email{clump@cs.mcgill.ca}

\author{J. Nelson Amaral}
%\authornote{Both authors contributed equally to this research.}
%\orcid{1234-5678-9012}
%\author{}
%\authornotemark[1]
%\email{}
\affiliation{%
  \institution{University of Alberta}
  \streetaddress{}
  \city{Edmonton}
  \state{Alberta}
  \country{Canada}  
}
\email{jamaral@ualberta.ca}

\author{Reid Copeland}
\affiliation{%
  \institution{IBM Toronto Laboratory}
  \streetaddress{}
  \city{Markham}
  \state{Ontario}
  \country{Canada}
}
\email{rcopelan@ca.ibm.com}

\author{Whitney T. Tsang}
\affiliation{%
  \institution{IBM Toronto Laboratory}
  \streetaddress{}
  \city{Markham}
  \state{Ontario}
  \country{Canada}
}
\email{whitneyt@ca.ibm.com}

%%
%% By default, the full list of authors will be used in the page
%% headers. Often, this list is too long, and will overlap
%% other information printed in the page headers. This command allows
%% the author to define a more concise list
%% of authors' names for this purpose.
%\renewcommand{\shortauthors}{Trovato and Tobin, et al.}

\newenvironment{pointform}{
  \begin{list}{$\bullet$}{
      \setlength{\topsep}{1mm}
      \setlength{\itemsep}{0mm}
      \setlength{\leftmargin}{2mm}
      \setlength{\rightmargin}{1mm}
      \setlength{\itemindent}{2mm}}}
  {\end{list}\vspace{1.5mm}}

%%
%% The abstract is a short summary of the work to be presented in the
%% article.
\begin{abstract}
  The compiler-driven performance workshop consisted of the presentation of reports on research progress at various academic and industrial sites across Canada and in the United States. Topics discussed in the workshop included, among others:
  \begin{pointform}
  \item innovative compiler analysis, transformation, and optimization techniques
  \item languages, compilers, and optimization techniques for multicore processors and other parallel architectures
  \item analysis and compilation techniques for dynamic programming languages
  \item compiling for streaming or heterogeneous hardware
  \item dynamic compilation for high-performance and real-time environments
  \item compilation and optimization for scripting languages
  \item compilation techniques for reducing power
  \item program safety
  \item whole system optimization and analysis
  \item tools and infrastructure for compiler research
\end{pointform}
\end{abstract}

%%
%% The code below is generated by the tool at http://dl.acm.org/ccs.cfm.
%% Please copy and paste the code instead of the example below.
%%
\begin{CCSXML}
<ccs2012>
<concept>
<concept_id>10011007.10011006.10011041</concept_id>
<concept_desc>Software and its engineering~Compilers</concept_desc>
<concept_significance>500</concept_significance>
</concept>
<concept>
<concept>
<concept_id>10010147.10010169</concept_id>
<concept_desc>Computing methodologies~Parallel computing methodologies</concept_desc>
<concept_significance>500</concept_significance>
</concept>
</ccs2012>
\end{CCSXML}

\ccsdesc{Software and its engineering~Compilers}
\ccsdesc{Computing methodologies~Parallel computing methodologies}

%%
%% Keywords. The author(s) should pick words that accurately describe
%% the work being presented. Separate the keywords with commas.
\keywords{compilers, optimization, parallelism, runtime systems, performance}

\maketitle

\section{Introduction}
This was the 18th edition of the Workshop on Compiler-Driven Performance (CDP) at CASCON. This workshop provides a unique opportunity for academic researchers, industry researchers, and developers from across Canada and the United States to examine the state-of-the-art compiler technology and discuss future directions for research and development.

This workshop addressed many issues of key relevance to CASCON. The end of processor frequency scaling has made compiler optimizations more important than ever. Varying programming models and dynamic runtimes have made it hard to understand system performance. The advent of multicore processors has made it difficult to achieve both high programmer productivity and performance. The CDP workshop is well suited to address these challenges.

\section{Format}
The workshop is structured as an all-day, multiple-speakers workshop. The workshop consists of several talks of 20--25 minutes each, with additional time for questions after each talk: each talk is followed by at least a 5-minute question-and-answer period, allowing the audience to discuss the presentation with the speaker and each other. Talks are grouped, by topic, into sessions of two or three talks, with either a half-hour break or lunch between sessions to provide further time for discussion and interaction. 

\section{Topics of Interest}
Developments in computing technology motivate a number of key challenges for compilers to address. The workshop has a particular focus on the following topics.

\subsection{Innovative Compiler Analysis, Transformation, and Optimization Techniques}

Today's software systems are often composed of several different languages and programming models. At the same time, the underlying processors and memory systems are typically complex, out-of-order, superscalar processors.  Managing this complexity, trying to produce efficient systems in a way that does not increase the burden on programmers, requires constant innovation in compilation methods and optimizations.

\subsection{Languages, Compilers, and Optimization Techniques for Multicore Processors and Other Parallel Architectures}

Highly-parallel hardware has created new opportunities for the utilization of such resources by advanced compilers. Progress has been reported both in the development of new analysis techniques, the exploitation of new technological solutions to facilitate the expression of concurrency and to deal with the needs for computation synchronization, e.g., hardware-supported transactional memory, and for data communication between computing domains. Approaches to improve the utilization of heterogeneous hardware platforms could also be discussed.

\subsection{Compiling for Streaming or Heterogeneous Hardware}

In addition to highly-parallel multiple-core processors for general-purpose and scientific computing, the computer hardware industry is also aggressively pursuing custom computing cores to accelerate key applications. Such heterogeneous computing systems were once limited to the embedded domain, but are becoming increasingly common for general-purpose computing. Examples include cores for encryption, compression, pattern-matching, and systems that have FPGA co-processors, graphics processing units (GPUs) and custom accelerators, which will likely soon be incorporated on-chip with regular processors. The resulting heterogeneous hardware presents another key challenge that the workshop target audience is working to address.

\subsection{Dynamic Compilation for High-Performance and Real-Time Environments}

Of ever-increasing importance are compilers that dynamically translate or optimize programs, not only to support interpreted languages such as Java, but also to exploit the run-time behaviour of programs written in C and C++ to improve efficiency and performance. Run-time adaptation can usually take advantage of more precise state information, allowing systems to present abstracted interfaces while ensuring an efficient implementation. Such layers of abstraction are important to allow programmers to efficiently target the emerging highly-parallel and heterogeneous hardware.

\subsection{Compilation, Optimization, and Analysis for Dynamic Languages}

The performance of scripting languages such as Python, Ruby, PHP, Javascript, and others, are of increasing importance to the overall performance of most online systems. These interpreted, and often dynamically-typed, languages pose many  challenges in terms of optimization design and analysis, requiring highly dynamic and adaptive techniques to overcome the lack of static information, language idioms, and novel workloads found in different execution contexts.

\subsection{Compilation Techniques for Reducing Power}

Reducing power consumption is a key challenge for all computer systems, from mobile devices through high-end supercomputers. Compilers that can optimize for power or coordinate the power-reduction features of other parts of the system are of great interest. This extends to the compiler itself, incorporating power-friendly methods in the context of dynamic compilation.

\subsection{Program Safety}

The size and complexity of many modern software projects makes programming errors both difficult to find and easy to produce. Compiler approaches have shown potential to improve code safety by detecting common bugs ahead of time or by automatically trapping more subtle errors at runtime. Such techniques are likely to play an increasing role in software development, lead to many analysis and runtime optimization challenges, and represent an interesting further application domain for software analysis.

\subsection{Whole System Optimization and Analysis}

Many applications are in practice run in a non-trivial context, along with other activities or programs. Individual program behaviours and resource competition then affect overall system performance. Designs that assess complementary or competitive behaviours, or that dynamically adjust individual execution to improve global system usage extend program optimization and analysis techniques to higher-level execution goals.

\subsection{Tools and Infrastructure for Compiler Research}

The changing technology landscape highlights the need for ever-improving compiler-based tools and infrastructure for understanding programs and performing research. Development of new analysis techniques and optimizations is facilitated by basic program exploration, profiling, and visualization, looking for further sources of semantic meaning that can be applied to improve performance, language design, or toward other optimization goals. 

\section{Schedule}
The workshop schedule included the following research talks:

\begin{enumerate}
\item \textit{Prithayan Barua}, \textit{Jun Shirako} and \textit{Whitney Tsang}. OMPSan: Static Verification of OpenMP's Data Mapping Constructs.
\item \textit{Guido Araujo}. Accelerating Task Parallelism using a RISC-V Engine information.
\item \textit{Prashanth Menon}, \textit{Todd Mowry} and \textit{Andrew Pavlo}. Regret-Free Compilation: An Optimization Framework for Recompiling SQL Queries on the Fly.
\item \textit{Hanfeng Chen}, \textit{Alexander Krolik}, \textit{Bettina Kemme}, \textit{Clark Verbrugge} and \textit{Laurie Hendren}. Improving Database Query Performance with Automatic Fusion.
\item \textit{Prabhjot Sandhu}, \textit{Clark Verbrugge} and \textit{Laurie Hendren}. A structure-driven performance analysis of sparse matrix-vector multiplication.
\item \textit{Jason Pizzuco}, \textit{Alexander Krolik} and \textit{Clark Verbrugge}. Sorting on GPUs.
\item \textit{Naser Ezzati-Jivan}, \textit{Wahab Hamou-Lhadj} and \textit{Michel Dagenais}. A detailed multilevel look at PHP compilation performance.
\end{enumerate}

\section{Summary}
Continued language and hardware development requires continued effort to identify new sources of potential optimization, and develop novel techniques for new contexts. The workshop on Compiler-Driven Performance allows researchers to share progress in developing new approaches to existing problems and in identifying new optimization opportunities and performance vectors for current and future languages. The workshop has the additional benefit of both providing and demonstrating strong cooperation between academia and industry---compiler optimization research is a highly practical domain, but also one in which cutting-edge research techniques can have direct application, and work in both academic and industrial contexts is improved through this level of cooperation.

\end{document}
\endinput
%%
%% End of file `sample-sigconf.tex'.
